\documentclass[]{article}
\usepackage{lmodern}
\usepackage{amssymb,amsmath}
\usepackage{ifxetex,ifluatex}
\usepackage{fixltx2e} % provides \textsubscript
\ifnum 0\ifxetex 1\fi\ifluatex 1\fi=0 % if pdftex
  \usepackage[T1]{fontenc}
  \usepackage[utf8]{inputenc}
\else % if luatex or xelatex
  \ifxetex
    \usepackage{mathspec}
  \else
    \usepackage{fontspec}
  \fi
  \defaultfontfeatures{Ligatures=TeX,Scale=MatchLowercase}
\fi
% use upquote if available, for straight quotes in verbatim environments
\IfFileExists{upquote.sty}{\usepackage{upquote}}{}
% use microtype if available
\IfFileExists{microtype.sty}{%
\usepackage{microtype}
\UseMicrotypeSet[protrusion]{basicmath} % disable protrusion for tt fonts
}{}
\usepackage[margin=1in]{geometry}
\usepackage{hyperref}
\hypersetup{unicode=true,
            pdftitle={Economics C142, Problem Set 3},
            pdfauthor={Vinay Maruri},
            pdfborder={0 0 0},
            breaklinks=true}
\urlstyle{same}  % don't use monospace font for urls
\usepackage{color}
\usepackage{fancyvrb}
\newcommand{\VerbBar}{|}
\newcommand{\VERB}{\Verb[commandchars=\\\{\}]}
\DefineVerbatimEnvironment{Highlighting}{Verbatim}{commandchars=\\\{\}}
% Add ',fontsize=\small' for more characters per line
\usepackage{framed}
\definecolor{shadecolor}{RGB}{248,248,248}
\newenvironment{Shaded}{\begin{snugshade}}{\end{snugshade}}
\newcommand{\AlertTok}[1]{\textcolor[rgb]{0.94,0.16,0.16}{#1}}
\newcommand{\AnnotationTok}[1]{\textcolor[rgb]{0.56,0.35,0.01}{\textbf{\textit{#1}}}}
\newcommand{\AttributeTok}[1]{\textcolor[rgb]{0.77,0.63,0.00}{#1}}
\newcommand{\BaseNTok}[1]{\textcolor[rgb]{0.00,0.00,0.81}{#1}}
\newcommand{\BuiltInTok}[1]{#1}
\newcommand{\CharTok}[1]{\textcolor[rgb]{0.31,0.60,0.02}{#1}}
\newcommand{\CommentTok}[1]{\textcolor[rgb]{0.56,0.35,0.01}{\textit{#1}}}
\newcommand{\CommentVarTok}[1]{\textcolor[rgb]{0.56,0.35,0.01}{\textbf{\textit{#1}}}}
\newcommand{\ConstantTok}[1]{\textcolor[rgb]{0.00,0.00,0.00}{#1}}
\newcommand{\ControlFlowTok}[1]{\textcolor[rgb]{0.13,0.29,0.53}{\textbf{#1}}}
\newcommand{\DataTypeTok}[1]{\textcolor[rgb]{0.13,0.29,0.53}{#1}}
\newcommand{\DecValTok}[1]{\textcolor[rgb]{0.00,0.00,0.81}{#1}}
\newcommand{\DocumentationTok}[1]{\textcolor[rgb]{0.56,0.35,0.01}{\textbf{\textit{#1}}}}
\newcommand{\ErrorTok}[1]{\textcolor[rgb]{0.64,0.00,0.00}{\textbf{#1}}}
\newcommand{\ExtensionTok}[1]{#1}
\newcommand{\FloatTok}[1]{\textcolor[rgb]{0.00,0.00,0.81}{#1}}
\newcommand{\FunctionTok}[1]{\textcolor[rgb]{0.00,0.00,0.00}{#1}}
\newcommand{\ImportTok}[1]{#1}
\newcommand{\InformationTok}[1]{\textcolor[rgb]{0.56,0.35,0.01}{\textbf{\textit{#1}}}}
\newcommand{\KeywordTok}[1]{\textcolor[rgb]{0.13,0.29,0.53}{\textbf{#1}}}
\newcommand{\NormalTok}[1]{#1}
\newcommand{\OperatorTok}[1]{\textcolor[rgb]{0.81,0.36,0.00}{\textbf{#1}}}
\newcommand{\OtherTok}[1]{\textcolor[rgb]{0.56,0.35,0.01}{#1}}
\newcommand{\PreprocessorTok}[1]{\textcolor[rgb]{0.56,0.35,0.01}{\textit{#1}}}
\newcommand{\RegionMarkerTok}[1]{#1}
\newcommand{\SpecialCharTok}[1]{\textcolor[rgb]{0.00,0.00,0.00}{#1}}
\newcommand{\SpecialStringTok}[1]{\textcolor[rgb]{0.31,0.60,0.02}{#1}}
\newcommand{\StringTok}[1]{\textcolor[rgb]{0.31,0.60,0.02}{#1}}
\newcommand{\VariableTok}[1]{\textcolor[rgb]{0.00,0.00,0.00}{#1}}
\newcommand{\VerbatimStringTok}[1]{\textcolor[rgb]{0.31,0.60,0.02}{#1}}
\newcommand{\WarningTok}[1]{\textcolor[rgb]{0.56,0.35,0.01}{\textbf{\textit{#1}}}}
\usepackage{graphicx,grffile}
\makeatletter
\def\maxwidth{\ifdim\Gin@nat@width>\linewidth\linewidth\else\Gin@nat@width\fi}
\def\maxheight{\ifdim\Gin@nat@height>\textheight\textheight\else\Gin@nat@height\fi}
\makeatother
% Scale images if necessary, so that they will not overflow the page
% margins by default, and it is still possible to overwrite the defaults
% using explicit options in \includegraphics[width, height, ...]{}
\setkeys{Gin}{width=\maxwidth,height=\maxheight,keepaspectratio}
\IfFileExists{parskip.sty}{%
\usepackage{parskip}
}{% else
\setlength{\parindent}{0pt}
\setlength{\parskip}{6pt plus 2pt minus 1pt}
}
\setlength{\emergencystretch}{3em}  % prevent overfull lines
\providecommand{\tightlist}{%
  \setlength{\itemsep}{0pt}\setlength{\parskip}{0pt}}
\setcounter{secnumdepth}{0}
% Redefines (sub)paragraphs to behave more like sections
\ifx\paragraph\undefined\else
\let\oldparagraph\paragraph
\renewcommand{\paragraph}[1]{\oldparagraph{#1}\mbox{}}
\fi
\ifx\subparagraph\undefined\else
\let\oldsubparagraph\subparagraph
\renewcommand{\subparagraph}[1]{\oldsubparagraph{#1}\mbox{}}
\fi

%%% Use protect on footnotes to avoid problems with footnotes in titles
\let\rmarkdownfootnote\footnote%
\def\footnote{\protect\rmarkdownfootnote}

%%% Change title format to be more compact
\usepackage{titling}

% Create subtitle command for use in maketitle
\newcommand{\subtitle}[1]{
  \posttitle{
    \begin{center}\large#1\end{center}
    }
}

\setlength{\droptitle}{-2em}

  \title{Economics C142, Problem Set 3}
    \pretitle{\vspace{\droptitle}\centering\huge}
  \posttitle{\par}
    \author{Vinay Maruri}
    \preauthor{\centering\large\emph}
  \postauthor{\par}
      \predate{\centering\large\emph}
  \postdate{\par}
    \date{February 14, 2019}

\usepackage{booktabs}
\usepackage{longtable}
\usepackage{array}
\usepackage{multirow}
\usepackage{wrapfig}
\usepackage{float}
\usepackage{colortbl}
\usepackage{pdflscape}
\usepackage{tabu}
\usepackage{threeparttable}
\usepackage{threeparttablex}
\usepackage[normalem]{ulem}
\usepackage{makecell}
\usepackage{xcolor}

\begin{document}
\maketitle

\begin{Shaded}
\begin{Highlighting}[]
\KeywordTok{library}\NormalTok{(dplyr)}
\end{Highlighting}
\end{Shaded}

\begin{verbatim}
## Warning: package 'dplyr' was built under R version 3.5.2
\end{verbatim}

\begin{verbatim}
## 
## Attaching package: 'dplyr'
\end{verbatim}

\begin{verbatim}
## The following objects are masked from 'package:stats':
## 
##     filter, lag
\end{verbatim}

\begin{verbatim}
## The following objects are masked from 'package:base':
## 
##     intersect, setdiff, setequal, union
\end{verbatim}

\begin{Shaded}
\begin{Highlighting}[]
\KeywordTok{library}\NormalTok{(ggplot2)}
\end{Highlighting}
\end{Shaded}

\begin{verbatim}
## Warning: package 'ggplot2' was built under R version 3.5.2
\end{verbatim}

\begin{Shaded}
\begin{Highlighting}[]
\KeywordTok{library}\NormalTok{(magrittr)}
\end{Highlighting}
\end{Shaded}

\begin{verbatim}
## Warning: package 'magrittr' was built under R version 3.5.2
\end{verbatim}

\begin{Shaded}
\begin{Highlighting}[]
\KeywordTok{library}\NormalTok{(reshape2)}
\end{Highlighting}
\end{Shaded}

\begin{verbatim}
## Warning: package 'reshape2' was built under R version 3.5.2
\end{verbatim}

\begin{Shaded}
\begin{Highlighting}[]
\KeywordTok{library}\NormalTok{(stargazer)}
\end{Highlighting}
\end{Shaded}

\begin{verbatim}
## Warning: package 'stargazer' was built under R version 3.5.2
\end{verbatim}

\begin{verbatim}
## 
## Please cite as:
\end{verbatim}

\begin{verbatim}
##  Hlavac, Marek (2018). stargazer: Well-Formatted Regression and Summary Statistics Tables.
\end{verbatim}

\begin{verbatim}
##  R package version 5.2.2. https://CRAN.R-project.org/package=stargazer
\end{verbatim}

\begin{Shaded}
\begin{Highlighting}[]
\KeywordTok{library}\NormalTok{(lubridate)}
\end{Highlighting}
\end{Shaded}

\begin{verbatim}
## Warning: package 'lubridate' was built under R version 3.5.2
\end{verbatim}

\begin{verbatim}
## 
## Attaching package: 'lubridate'
\end{verbatim}

\begin{verbatim}
## The following object is masked from 'package:base':
## 
##     date
\end{verbatim}

\begin{Shaded}
\begin{Highlighting}[]
\KeywordTok{library}\NormalTok{(lmtest)}
\end{Highlighting}
\end{Shaded}

\begin{verbatim}
## Warning: package 'lmtest' was built under R version 3.5.2
\end{verbatim}

\begin{verbatim}
## Loading required package: zoo
\end{verbatim}

\begin{verbatim}
## Warning: package 'zoo' was built under R version 3.5.2
\end{verbatim}

\begin{verbatim}
## 
## Attaching package: 'zoo'
\end{verbatim}

\begin{verbatim}
## The following objects are masked from 'package:base':
## 
##     as.Date, as.Date.numeric
\end{verbatim}

\begin{Shaded}
\begin{Highlighting}[]
\KeywordTok{library}\NormalTok{(ivpack)}
\end{Highlighting}
\end{Shaded}

\begin{verbatim}
## Warning: package 'ivpack' was built under R version 3.5.2
\end{verbatim}

\begin{verbatim}
## Loading required package: AER
\end{verbatim}

\begin{verbatim}
## Warning: package 'AER' was built under R version 3.5.2
\end{verbatim}

\begin{verbatim}
## Loading required package: car
\end{verbatim}

\begin{verbatim}
## Warning: package 'car' was built under R version 3.5.2
\end{verbatim}

\begin{verbatim}
## Loading required package: carData
\end{verbatim}

\begin{verbatim}
## Warning: package 'carData' was built under R version 3.5.2
\end{verbatim}

\begin{verbatim}
## 
## Attaching package: 'car'
\end{verbatim}

\begin{verbatim}
## The following object is masked from 'package:dplyr':
## 
##     recode
\end{verbatim}

\begin{verbatim}
## Loading required package: sandwich
\end{verbatim}

\begin{verbatim}
## Warning: package 'sandwich' was built under R version 3.5.2
\end{verbatim}

\begin{verbatim}
## Loading required package: survival
\end{verbatim}

\begin{verbatim}
## Warning: package 'survival' was built under R version 3.5.2
\end{verbatim}

\begin{Shaded}
\begin{Highlighting}[]
\KeywordTok{library}\NormalTok{(kableExtra)}
\end{Highlighting}
\end{Shaded}

\begin{verbatim}
## Warning: package 'kableExtra' was built under R version 3.5.2
\end{verbatim}

\begin{Shaded}
\begin{Highlighting}[]
\CommentTok{#importing data}
\NormalTok{data_raw <-}\StringTok{ }\KeywordTok{read.csv}\NormalTok{(}\StringTok{"/Users/EndlessWormhole/Desktop/Spring 2019/Econ C142/problem set 3/ovb.csv"}\NormalTok{)}
\end{Highlighting}
\end{Shaded}

\begin{Shaded}
\begin{Highlighting}[]
\CommentTok{#split the data into male and female groups.}
\NormalTok{female <-}\StringTok{ }\NormalTok{data_raw }\OperatorTok\StringTok{ }\NormalTok{dplyr}\OperatorTok{::}\KeywordTok{filter}\NormalTok{(female }\OperatorTok{==}\StringTok{ }\DecValTok{1}\NormalTok{)}
\end{Highlighting}
\end{Shaded}

\begin{verbatim}
## Warning: package 'bindrcpp' was built under R version 3.5.2
\end{verbatim}

\begin{Shaded}
\begin{Highlighting}[]
\NormalTok{male <-}\StringTok{ }\NormalTok{data_raw }\OperatorTok\StringTok{ }\NormalTok{dplyr}\OperatorTok{::}\KeywordTok{filter}\NormalTok{(female }\OperatorTok{==}\StringTok{ }\DecValTok{0}\NormalTok{)}
\end{Highlighting}
\end{Shaded}

\begin{Shaded}
\begin{Highlighting}[]
\CommentTok{#running model 1: constant, immigrant status on logwage. }
\CommentTok{#first females}
\NormalTok{model1female <-}\StringTok{ }\KeywordTok{lm}\NormalTok{(logwage }\OperatorTok{~}\StringTok{ }\NormalTok{imm, }\DataTypeTok{data =}\NormalTok{ female)}
\end{Highlighting}
\end{Shaded}

\begin{Shaded}
\begin{Highlighting}[]
\CommentTok{#next model 1 for males}
\NormalTok{model1male <-}\StringTok{ }\KeywordTok{lm}\NormalTok{(logwage }\OperatorTok{~}\StringTok{ }\NormalTok{imm, }\DataTypeTok{data =}\NormalTok{ male)}
\end{Highlighting}
\end{Shaded}

\begin{Shaded}
\begin{Highlighting}[]
\CommentTok{#the next model we will run is model 2: constant and education on logwage}
\CommentTok{#first females again}
\NormalTok{model2female <-}\StringTok{ }\KeywordTok{lm}\NormalTok{(logwage }\OperatorTok{~}\StringTok{ }\NormalTok{educ, }\DataTypeTok{data =}\NormalTok{ female)}
\end{Highlighting}
\end{Shaded}

\begin{Shaded}
\begin{Highlighting}[]
\CommentTok{#then males again}
\NormalTok{model2male <-}\StringTok{ }\KeywordTok{lm}\NormalTok{(logwage }\OperatorTok{~}\StringTok{ }\NormalTok{educ, }\DataTypeTok{data =}\NormalTok{ male)}
\end{Highlighting}
\end{Shaded}

\begin{Shaded}
\begin{Highlighting}[]
\CommentTok{#now we run model 3: constant and education on immigrant status.}
\CommentTok{#first females}
\NormalTok{model3female <-}\StringTok{ }\KeywordTok{lm}\NormalTok{(imm }\OperatorTok{~}\StringTok{ }\NormalTok{educ, }\DataTypeTok{data =}\NormalTok{ female)}
\end{Highlighting}
\end{Shaded}

\begin{Shaded}
\begin{Highlighting}[]
\CommentTok{#then males}
\NormalTok{model3male <-}\StringTok{ }\KeywordTok{lm}\NormalTok{(imm }\OperatorTok{~}\StringTok{ }\NormalTok{educ, }\DataTypeTok{data =}\NormalTok{ male)}
\end{Highlighting}
\end{Shaded}

\begin{Shaded}
\begin{Highlighting}[]
\CommentTok{#next we will run model 4: constant and immmigrant status on education}
\CommentTok{#first females}
\NormalTok{model4female <-}\StringTok{ }\KeywordTok{lm}\NormalTok{(educ }\OperatorTok{~}\StringTok{ }\NormalTok{imm, }\DataTypeTok{data =}\NormalTok{ female)}
\end{Highlighting}
\end{Shaded}

\begin{Shaded}
\begin{Highlighting}[]
\CommentTok{#then males}
\NormalTok{model4male <-}\StringTok{ }\KeywordTok{lm}\NormalTok{(educ }\OperatorTok{~}\StringTok{ }\NormalTok{imm, }\DataTypeTok{data =}\NormalTok{ male)}
\end{Highlighting}
\end{Shaded}

\begin{Shaded}
\begin{Highlighting}[]
\CommentTok{#finally we will run model 5: constant, immigrant status, and education on logwage}
\CommentTok{#first females}
\NormalTok{model5female <-}\StringTok{ }\KeywordTok{lm}\NormalTok{(logwage }\OperatorTok{~}\StringTok{ }\NormalTok{educ }\OperatorTok{+}\StringTok{ }\NormalTok{imm, }\DataTypeTok{data =}\NormalTok{ female)}
\end{Highlighting}
\end{Shaded}

\begin{Shaded}
\begin{Highlighting}[]
\CommentTok{#then males}
\NormalTok{model5male <-}\StringTok{ }\KeywordTok{lm}\NormalTok{(logwage }\OperatorTok{~}\StringTok{ }\NormalTok{educ }\OperatorTok{+}\StringTok{ }\NormalTok{imm, }\DataTypeTok{data =}\NormalTok{ male)}
\end{Highlighting}
\end{Shaded}

\begin{Shaded}
\begin{Highlighting}[]
\CommentTok{#I am choosing to report my output in a stargazer table for conciseness of code and the submission document. Also I think it }
\CommentTok{#reports the coefficients and assorted test statistics more nicely.}
\KeywordTok{stargazer}\NormalTok{(model1female, model2female, model3female, model4female, model5female, }\DataTypeTok{type =} \StringTok{"latex"}\NormalTok{, }\DataTypeTok{title =} \StringTok{"Estimation of the 5 models specified in Question 2(a) for Females"}\NormalTok{, }\DataTypeTok{header =} \OtherTok{FALSE}\NormalTok{, }\DataTypeTok{multicolumn =} \OtherTok{FALSE}\NormalTok{, }\DataTypeTok{column_sep_width =} \StringTok{'0.1pt'}\NormalTok{, }\DataTypeTok{single_row =} \OtherTok{TRUE}\NormalTok{)}
\end{Highlighting}
\end{Shaded}

\begin{verbatim}
## 
## \begin{table}[!htbp] \centering 
##   \caption{Estimation of the 5 models specified in Question 2(a) for Females} 
##   \label{} 
## \begin{tabular}{@{\extracolsep{5pt}}lccccc} 
## \\[-1.8ex]\hline 
## \hline \\[-1.8ex] 
##  & \multicolumn{5}{c}{\textit{Dependent variable:}} \\ 
## \cline{2-6} 
## \\[-1.8ex] & logwage & logwage & imm & educ & logwage \\ 
## \\[-1.8ex] & (1) & (2) & (3) & (4) & (5)\\ 
## \hline \\[-1.8ex] 
##  imm & $-$0.180$^{***}$ &  &  & $-$1.492$^{***}$ & $-$0.010 \\ 
##   & (0.017) &  &  & (0.067) & (0.015) \\ 
##   & & & & & \\ 
##  educ &  & 0.114$^{***}$ & $-$0.030$^{***}$ &  & 0.114$^{***}$ \\ 
##   &  & (0.002) & (0.001) &  & (0.002) \\ 
##   & & & & & \\ 
##  Constant & 2.886$^{***}$ & 1.235$^{***}$ & 0.607$^{***}$ & 14.452$^{***}$ & 1.241$^{***}$ \\ 
##   & (0.007) & (0.030) & (0.019) & (0.029) & (0.031) \\ 
##   & & & & & \\ 
## \hline \\[-1.8ex] 
## Observations & 10,601 & 10,601 & 10,601 & 10,601 & 10,601 \\ 
## R$^{2}$ & 0.011 & 0.224 & 0.044 & 0.044 & 0.224 \\ 
## Adjusted R$^{2}$ & 0.011 & 0.224 & 0.044 & 0.044 & 0.224 \\ 
## Residual Std. Error & 0.664 (df = 10599) & 0.588 (df = 10599) & 0.381 (df = 10599) & 2.706 (df = 10599) & 0.588 (df = 10598) \\ 
## F Statistic & 118.530$^{***}$ (df = 1; 10599) & 3,057.517$^{***}$ (df = 1; 10599) & 490.608$^{***}$ (df = 1; 10599) & 490.608$^{***}$ (df = 1; 10599) & 1,528.907$^{***}$ (df = 2; 10598) \\ 
## \hline 
## \hline \\[-1.8ex] 
## \textit{Note:}  & \multicolumn{5}{r}{$^{*}$p$<$0.1; $^{**}$p$<$0.05; $^{***}$p$<$0.01} \\ 
## \end{tabular} 
## \end{table} 
## 
## \begin{table}[!htbp] \centering 
##   \caption{Estimation of the 5 models specified in Question 2(a) for Females} 
##   \label{} 
## \begin{tabular}{@{\extracolsep{5pt}} c} 
## \\[-1.8ex]\hline 
## \hline \\[-1.8ex] 
## 0.1pt \\ 
## \hline \\[-1.8ex] 
## \end{tabular} 
## \end{table} 
## 
## \begin{table}[!htbp] \centering 
##   \caption{Estimation of the 5 models specified in Question 2(a) for Females} 
##   \label{} 
## \begin{tabular}{@{\extracolsep{5pt}} c} 
## \\[-1.8ex]\hline 
## \hline \\[-1.8ex] 
## TRUE \\ 
## \hline \\[-1.8ex] 
## \end{tabular} 
## \end{table}
\end{verbatim}


\end{document}
